\section{Discussion}
\label{sec:discuss} 

Large-scale file systems are complex to architect and deploy and they are
equally complex to operate. Key to successful deployment and operations is
understanding the I/O workloads exercising these file systems. At the OLCF we
pay particular attention to collecting file system statistics for better
understanding how our file systems are used. Over the 5 year lifetime of the
Spider 1 file system, we collected detailed operational data and we are doing
the same with the Spider 2 file system. Spider 1 data was essential in
architecting the Spider 2 system, and we are expecting that Spider 2 data will
instrumental in building the next-generation parallel file system at the OLCF.
 

As the Spider 2 data shows us, we observed roughly 75\% writes on the Spider 2
storage system. This is slightly more different than what we have observed on
Spider 1 and we conclude that Spider 2 is currently exercised with more
write-heavy I/O workloads. At the same time our write requests consist of only
4kB and 1MB write requests. This is a good step in the right direction, since
the 512kB write requests have been eliminated from the system due to the fixes 
in dm-multipath, I/O request scheduling and ib_srp kernel module feature 
development and bug-fixing. The existence of extra 512kB was detrimental to the
overall aggregate I/O performance on Spider 1 and this has been corrected in
Spider 2.  
