\section*{Conclusions}
\label{sec:conc}

Large-scale file systems are complex to architect and deploy and are equally
complex to operate. Key to successful deployment and operation is understanding
the I/O workloads exercising these file systems. At the OLCF we continue to pay
particular attention to collecting file system statistics for better
understanding how our file systems are used. Over the 5-year lifetime of Spider
1, we collected detailed operational data and we are doing the same with the
Spider 2. Gathering, curating, and querying this data was essential in
architecting the Spider 2 system, and we are expecting this effort will again
be instrumental in building the next-generation parallel file system at the
OLCF.


As the Spider 2 data shows us, we observed roughly 75\% writes on the Spider 2
storage system. This is slightly more different than what we have observed on
Spider 1 and we conclude that Spider 2 is currently exercised with more
write-heavy I/O workloads. At the same time our write requests consist of only
4kB and 1MB write requests. This is a good step in the right direction, since
the 512kB write requests have been eliminated from the system due to the fixes
in dm-multipath, I/O request scheduling and ib\_srp kernel module feature
development and bug-fixing. The existence of extra 512kB was detrimental to the
overall aggregate I/O performance on Spider 1 and this has been corrected in
Spider 2. As it is evident from the observed data, we also believe a Burst
Buffer layer between our compute platforms and the parallel file system will
better serve our scientific workloads by absorbing and aligning the bursty I/O
patterns and therefore improving the application I/O performances. 
